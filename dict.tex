\documentclass[a4paper,12pt,fleqn]{scrartcl}
\usepackage[l2tabu,orthodox]{nag}% Old habits die hard. All the same, there are commands, classes and packages which are outdated and superseded. nag provides routines to warn the user about the use of those.

\usepackage[all,error]{onlyamsmath}% Error on deprecated math commands like $$ $$.
\usepackage[strict=true]{csquotes}

%\usepackage{color}

\usepackage{listings}
\lstset{frame=single,framerule=0pt,language={C},showstringspaces=false,numbers=left,columns=fullflexible}

 
% COMP2111-specific macros. See
% http://www.cse.unsw.edu.au/~cs2111/18s1/LaTeX/primer.html
\usepackage{2111defs2}
\usepackage[colorlinks=true]{hyperref}

\newcommand{\assn}[1]{{\color{red}\left\{#1\right\}}}
\newcommand{\remark}[1]{{\sffamily\color{blue}{#1}}}

% define some convenience macros specific to this task
\newcommand{\perm}{\mathsf{perm}}
%title variable
\title{Assignment 3}
\author{Ruofei HUANG(z5141448)\and
Anqi ZHU(z5141541)
}

% custom command for ass3
\newcommand{\domt}{\textbf{dom}(t)}
\newcommand{\WORD}{\textbf{word }}

% some old custom command
\newcommand{\variant}[3]{(#1:#2\mapsto #3)}
\newcommand{\oldwhatever}[4]{\variant{#1}{#2}{\variant{#1[#2]}{#3}{#4}}}
\newcommand{\whatever}[4]{\variant{#1}{#2\mapsto #3}{#4}}
\newcommand{\ah}{\mathsf{a}}
\newcommand{\be}{\mathsf{b}}
\newcommand{\length}[1]{\left|#1\right|}
\newcommand{\noof}[2]{\left\|#1\right\|_{#2}}

\newcommand{\pre}{\mathit{pre}}
\newcommand{\post}{\mathit{post}}
\def\L{\mathcal{L}}
\begin{document}
\maketitle
\section{Task 1}
\subsection{Some useful symbol definitions about manipulating $\textbf{word}$s }
A $\textbf{word}$, as mentioned in the assignment specification, is defined as a 
finite sequence of letters over $L = {'a', .., 'z'}$. The specification already 
defines the relationship of '$\leq$' between a $\textbf{word}\ v$ and a 
$\textbf{word}\ w$ (which means $v$ is a prefix of $w$ or $w$ itself if 
$v\leq w$). 
\\\\
So we would like to further this definition and define a relationship of '$<$' 
when $v$ is a proper prefix of $w$ which cannot be $w$ itself if $v < w$. That 
means:
\begin{align*}
    v < w \Equiv v\leq w \And v\neq w
\end{align*}
We also would like to define a symbol $|w|$ that represents the length of a 
$word\ w$.We formally define this by:
\begin{align*}
    |w| = 
    \begin{cases}
        0 & \text{if } w = \epsilon\\
        1 + |w'| & \text{else }
    \end{cases}\\\\
    where\ \exists l\in L\ \{w = w'\ l\}
\end{align*} 
\subsection{Syntactic(Abstract) Data Type $Dict$ }
% TODO: init^{Dict} is missing?
Inspired by the program sketch and the assignment statement, we could describe 
the syntactic data type $Dict$ as below (the encapsulated state would be a 
dictionary word set $W$).
\begin{align*}
    &Dict = 
    (
        W = \phi,  \\
        &\left( 
            \begin{array}{l}
                \PROC\ addword^{Dict}(\WORD w)\cdot 
                b, W:[\True, b = b_0\And W = W_0\cup \{w\}]\\
                \FUNC\ checkword^{Dict}(\WORD w):\\
                    \qquad \mathbb{B}\cdot
                    \VAR\ b\cdot\ b, W:[\True, b = (w\in W)\And W = W_0];\ 
                    \RETURN\ b\\
                \PROC\ delword^{Dict}(\WORD w)\cdot
                b, W:[ w\in W, b = b_0\And W = W_0 \backslash \{w\}]
            \end{array}
        \right)
    )
\end{align*}
\section{Task 2}
% \subsection{Data Invariant} 
% \begin{gather*}
%     \All{w \in \domt,t(w)= 1, w' \leq w}{w' \in \domt}
% \end{gather*}
\subsection{Data Type Refinement}
Now we would like to refine $Dict$ to a second data type $DictA$ where we 
replace $W$ with a trie $t$. We would also like to define the domain of a $t$ as
$\domt$. We shall use this definition later in our refinement.
\subsubsection{Inductive Relation Predicate}
The correspondence between the two state space $W$ and $t$ is captured by the 
inductively defined predicate. 
\begin{gather*}
    r = (W = 
    \{ 
        w \in \domt| t(w)=1
    \})
\end{gather*}
which we can translate into a relation function that transfers a concrete 
state space $t$ to an abstract state space $W$:
\begin{gather*}
    f(t) = 
    \left\{
        w \in  \domt | t(w) = 1
    \right\}
\end{gather*}
With that in mind, we can propose the initialisation predicate and corresponding
operations of $DictA$.
\subsubsection{Initialisation Predicate}
We would like to define the initialisation predicate of DictA as follows:
\begin{gather*}
    init^{DictA} = (t\Ass\{\epsilon\mapsto 0\}) 
\end{gather*}
\subsubsection{Operations}
We would like to define the operations of DictA as follows:
\begin{gather*}
    \PROC\ addword^{DictA}(\WORD w)\cdot b, t:\\
        \qquad [ \ \True, b = b_0\And \\
        \qquad\qquad t = \left( 
                \begin{array}{l}
                    w\notin \domt\And t = {t_0\cup\{w\mapsto 1\}\cup\{w' < w\And 
                    w'\notin \domt | w'\mapsto 0\}}\\
                    w\in\domt\And t = (t_0:w\mapsto 1)\\
                \end{array}
            \right)]\\
    \FUNC\ checkword^{DictA}(\WORD w):\mathbb{B}\cdot
        \VAR\ b\cdot b, t:\\
        \qquad[\True, t = t_0\And b = (w\in \domt\And t(w)=1)];\ \RETURN\ b\\
    \PROC\ delword^{DictA}(\WORD w)\cdot b, t:\\
        \qquad[\True , b = b_0\And
        (w \notin \domt \OR t\Ass(t: w \mapsto 0))]\\ 
\end{gather*} 
\subsection{Proof of Refinement}
Now we would like to start proving the refinements of the initialization and 
each operation from $t$ to $W$. \\\\
We start from proving the refinement between $init^{DictA}$ and $init^{Dict}$.
\begin{align*}
    &init^{DictA} \Implies init^{Dict}\subst{f(t)}{W}\\
    \justification{Definition of $init^{DictA}$ and $init^{Dict}$}
    & \All {w \in \domt }{t(w)= 0} \Implies W = \phi
\end{align*}
Since all our precondition of concrete is trivial which all of them are \True, 
we don't need to proof the condition ($3_f$).But condition ($4_f$) must be 
checked for all three operations.\\
For the $addword$ we proof:
\begin{align*}
    &pre_{addword^{Dict}}\subst{f(t_0)}{W}\And post_{addword^{DictA}}\\
    \justification{Definition of $addword^{Dict}$ and $addword^{DictA}$ }
    % 
    & \True\subst{f(t_0)}{W} \And 
        b = b_0\And t = t_0\cup \{w\mapsto 1\}\cup 
        \{
            w' < w \And w' \notin\domt | w' \mapsto 0
        \}\\
    %
    \justification[\Implies]{Definition of $f$ }
    &f(t)= f( t_0\cup \{w\mapsto 1\}\cup 
    \{
        w' < w \And w' \notin\domt | w'\mapsto 0
    \})\\
    \justification{Logic} 
    &f(t)= f(t_0)\cup \{w\}\\ 
    \justification{Definition of $addword^{Dict}$ and $addword^{DictA}$ }
    &post_{addword^{Dict}}\subst{f(t_0),f(t)}{W_0, W}
\end{align*}
For the $checkword$ we proof:
\begin{align*}
    &pre_{checkword^{Dict}}\subst{f(t_0)}{W}\And post_{checkword^{DictA}}\\
    \justification{Definition of $checkword^{DictA}$ and $checkword^{Dict}$}
    &\True\subst{f(t_0)}{W} \And b = (w\in \domt\And t = t_0 )\\
    %
    \justification[\Implies]{Definition of $f$}
    &b = (w \in f(t)\And t(w)= 1)\And t = t_0\\
    \justification{Definition of $checkword^{DictA}$ and $checkword^{Dict}$}
    & post_{checkword^{Dict}}\subst{f(t_0),f(t)}{W_0,W}
\end{align*}
For the $delword$ we proof:
\begin{align*}
    &pre_{delword^{Dict}}\subst{f(t_0)}{W}\And post_{delword^{DictA}}\\
    \justification{Definition of $delword^{DictA}$ and $delword^{Dict}$}
    &w\in f(t_0) \And b = b_0\And
        (w \notin \domt \OR t\Ass t: w \mapsto 0)\\
    \justification[\Implies]{Definition of $f$}
    &f(t) = f(t_0)\backslash\{w\}\\
    \justification{Definition of $delword^{DictA}$ and $delword^{Dict}$}
    &post_{delword^{Dict}}\subst{f(t_0),f(t)}{W_0,W}
\end{align*}

\section{Task 3}
We derive our code in to five parts by $init$, data type's operations and a 
$popWord$ for recursive calls. 
\subsection{init}
From the spec we have:
\begin{align*}
    &\domt =\{\epsilon\} \And f(t)= \phi\\
    \refstep{ass}{t \Ass \{\epsilon \mapsto 0\}}
\end{align*}

\subsection{popWord}
A popWord is for us to have a easily use recursive calls, it is also a bridge 
between C and our Toy language\footnote{This fucntion would not
appear in our C code, the pointer to the node would be solve this problem.And 
pointer is a difficult part for Toy Language to express and proof.}. The purpose
of this function is to pop the first given letters of a given word. 
\subsubsection{Math Function}
\begin{gather*}
    POPWORD(w,i) = w' 
    \\
    where \  w,w' \in \WORD \And i \in \nat \And w' \leq w \And |w'| = i
\end{gather*}

\subsubsection{Toy Language Function}
\begin{gather*}
    \FUNC\ popWord(\VALUE\ w, \VALUE\ i )\cdot\\
    \qquad \nt{
        \VAR\ w' \cdot w':[
            i \leq |w|,
            w'\leq w \And |w'| = i
        ]; \RETURN\ w'  
    }{(P1)}
\end{gather*}
% TODO: derive code for popWord 


\subsection{addword}
From the spec\footnote{Definition of this is in the Assignment 3 requirements of
cs2111.} we have:
\begin{gather*}
    \PROC\ addword^{DictA}(\VALUE\ w)\cdot \\
        \qquad
        \nt{
            b, t:[\True, b = b_0\And t = t_0\cup \{w\mapsto 1\}\cup 
            \{
                w' < w \And w' \notin \domt | w' \mapsto 0
            \}]
        }{(A1)}
\end{gather*}
\begin{align*}
    \lrefstep{(A1)}{c-frame}{
        t:[\True,t = t_0\cup \{w\mapsto 1\}\cup 
        \{
            w' < w \And w' \notin \domt | w' \mapsto 0
        \}]
    }
\end{align*}
\subsection{checkword}
From the spec we have:
\begin{gather*}
    \FUNC\ checkword^{DictA}(\VALUE\ w)\\
        \qquad
        \nt{
            \mathbb{B}\cdot \VAR\ b\cdot b, t:
            [\True, b = (w\in \domt\And t = t_0)];\ \RETURN\ b
        }{(C1)}
\end{gather*}
\begin{align*}
    \lrefstep{(C1)}{c-frame}{
        \mathbb{B}\cdot \VAR\ b\cdot b:
        [\True, b = (w\in \domt)];\ \RETURN\ b
    }\\
    \refstep{proc, $ 0 \leq |w|$ }
    {
        \RETURN\ doCheckword(w, 0);
    }
\end{align*}
Where we define a recursive procedure call to do the dirty work also align the 
pre and post condition: 
\begin{gather*}
    \FUNC\ doCheckword(\VALUE\ w, \VALUE\ index:\nat )\\
        \qquad
        \nt{
            \mathbb{B}\cdot \VAR\ b\cdot b, \VAR\ index
            [index \leq w, b = (w\in \domt\And t(w)=1)];\ \RETURN\ b
        }{(C2)}
\end{gather*}
\begin{align*}
    \lrefstep{(C2)}{if}{
        \IF ~ index < |w| ~ \\ 
        &\THEN 
        \nt{
            b,indexp[ index < |w|\And \pre{(C2)}, \post{(C2)}]
        }{(C3-1)} \\
        &\ELSE 
        \nt{
            b,index[ index \geq |w| \And \pre{(C2)}, \post{(C2)}]
        }{(C3-2)}\\
        &\FI\\
    }
    % TODO: try to use an early return, but there is not in Toy languague
    %       How to check whether the prefix is in the dom(t) 
    \lrefstep{(C3-1)}{if}{
        \IF ~(POPWORD(w,index)\in \domt)~\\
        &\THEN \\ 
        &\ELSE \\
        &\FI \\
    }
    \lrefstep{(C3-1)}{func}{
        \RETURN\ doCheckword(w, index +1);
    }\\
    \lrefstep{(C3-2)}{$index \geq |w| \And index \leq |w| \Implies 
    index = |w|$ and definition of $\post{(C2)}$}{
        \RETURN\ (w\in \domt \And t(w) = 1)
    }
\end{align*}
\subsection{delword}
From the spec we have:
\begin{gather*}
    \PROC\ delword^{DictA}(\VALUE\ w)\\
        \qquad
        \nt{
            \cdot b, t:
            [\True , b = b_0\And
            (w \notin \domt \OR t\Ass t: w \mapsto 0)]
        }{(D1)}    
\end{gather*}


\end{document}