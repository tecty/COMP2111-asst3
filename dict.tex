\documentclass[a4paper,12pt,fleqn]{scrartcl}
\usepackage[l2tabu,orthodox]{nag}% Old habits die hard. All the same, there are commands, classes and packages which are outdated and superseded. nag provides routines to warn the user about the use of those.

\usepackage[all,error]{onlyamsmath}% Error on deprecated math commands like $$ $$.
\usepackage[strict=true]{csquotes}

%\usepackage{color}

\usepackage{listings}
\lstset{frame=single,framerule=0pt,language={C},showstringspaces=false,numbers=left,columns=fullflexible}


% COMP2111-specific macros. See
% http://www.cse.unsw.edu.au/~cs2111/18s1/LaTeX/primer.html
\usepackage{2111defs2}
\usepackage[colorlinks=true]{hyperref}

\newcommand{\assn}[1]{{\color{red}\left\{#1\right\}}}
\newcommand{\remark}[1]{{\sffamily\color{blue}{#1}}}

% define some convenience macros specific to this task
\newcommand{\perm}{\mathsf{perm}}
%title variable
\title{Assignment 3}
\author{Ruofei HUANG(z5141448)\and
Anqi ZHU(z5141541)
}

% custom command for ass3
\newcommand{\domt}{\textbf{dom}(t)}
\newcommand{\WORD}{\textbf{word }}

% some old custom command
\newcommand{\variant}[3]{(#1:#2\mapsto #3)}
\newcommand{\oldwhatever}[4]{\variant{#1}{#2}{\variant{#1[#2]}{#3}{#4}}}
\newcommand{\whatever}[4]{\variant{#1}{#2\mapsto #3}{#4}}
\newcommand{\ah}{\mathsf{a}}
\newcommand{\be}{\mathsf{b}}
\newcommand{\length}[1]{\left|#1\right|}
\newcommand{\noof}[2]{\left\|#1\right\|_{#2}}

\newcommand{\pre}{\mathit{pre}}
\newcommand{\post}{\mathit{post}}
\def\L{\mathcal{L}}
\begin{document}
\maketitle
\section{Task 1}
\subsection{Type $word$ }
This definiation of wrod is basicly from requirement of assignment 3
% TODO: Add a more definition here. 
We say two words $ v,w$ that v is an absolute prefix of w as $v < w$ which is 
define as $ v \leq w \And v \neq w $.
\subsection{abstract Data Type $Dict$ }
% TODO: init^{Dict} is missing?
According to the specified problem statement in the assignment, 
we could describe the syntactic data type $Dict$ as below.
The encapsulated state is a dictionary word set $W$.
\begin{align*}
    &Dict = 
    (
        W = \phi,  \\
        &\left( 
            \begin{array}{l}
                \PROC\ addword^{Dict}(\WORD w)\cdot 
                b, W:[\True, b = b_0\And W = W_0\cup \{w\}]\\
                \FUNC\ checkword^{Dict}(\WORD w):\mathbb{B}\cdot
                    \VAR\ b\cdot\ b, W:[\True, b = (w\in W_0)];\ 
                    \RETURN\ b\\
                \PROC\ delword^{Dict}(\WORD w)\cdot
                b, W:[ w\in W, b = b_0\And W = W_0 \backslash \{w\}]
            \end{array}
        \right)
    )
\end{align*}
\section{Task 2}
Now we would like to refine $Dict$ to a second data type $DictA$ where we 
replace $W$ with a trie $t$,the corresponding trie domain $D=\domt$.
It represents the set of all tries according to the domain. 
We shall use this definition later in our refinement.\\

\subsection{Datat Invariant} 
\begin{gather*}
    \All{w \in \domt,t(w)= 1, w' \leq w}{w' \in \domt}
\end{gather*}


\subsection{Data Type Refinement}
This suggests we should first build up a inductively defined predicate to ensure
 the provable relations between
$DictA$ and $Dict$.
\begin{gather*}
    r = (W = 
    \{ 
        w \in \domt| t(w)=1
    \})
\end{gather*}
which we can translate into a function from concrete to abstract values:
\begin{gather*}
    f(t) = 
    \left\{
        w \in  \domt | t(w) = 1
    \right\}
\end{gather*}
With that in mind we propose the initialisation predicate 
$init^{DictA} = (\domt =\{\epsilon\} \And f(t)= \phi )$ and operations given as follows.
\begin{gather*}
    \PROC\ addword^{DictA}(\WORD w)\cdot b, t:\\
        \qquad [ \ \True, b = b_0\And t = t_0\cup \{w\mapsto 1\}\cup
        \{
            w' < w \And w' \notin \domt | w' \mapsto 0
        \}]\\
    \FUNC\ checkword^{DictA}(\WORD w) \mathbb{B}\cdot
        \VAR\ b\cdot b, t:\\
        \qquad[\True, b = (w\in \domt\And t = t_0)];\ \RETURN\ b\\
    \PROC\ delword^{DictA}(\WORD w)\cdot b, t:\\
        \qquad[\True , b = b_0\And
        (w \notin \domt \OR t\Ass t: w \mapsto 0)]\\ 
\end{gather*} 
\subsection{Proof of Refinement}
We need start the proof with the initialisation:
\begin{align*}
    &init^{DictA} \Implies init^{Dict}\subst{f(t)}{W}\\
    \justification{Definition of $init^{DictA}$ and $init^{Dict}$}
    & \All {w \in \domt }{t(w)= 0} \Implies W = \phi
\end{align*}
Since all our precondition of concrete is trivial which all of them are \True, 
we don't need to proof the condition ($3_f$).But condition ($4_f$) must be 
checked for all three operations.\\
For the $addword$ we proof:
\begin{align*}
    &pre_{addword^{Dict}}\subst{f(t_0)}{W}\And post_{addword^{DictA}}\\
    \justification{Definition of $addword^{Dict}$ and $addword^{DictA}$ }
    % 
    & \True\subst{f(t_0)}{W} \And 
        b = b_0\And t = t_0\cup \{w\mapsto 1\}\cup 
        \{
            w' < w \And w' \notin\domt | w' \mapsto 0
        \}\\
    %
    \justification[\Implies]{Definition of $f$ }
    &f(t)= f( t_0\cup \{w\mapsto 1\}\cup 
    \{
        w' < w \And w' \notin\domt | w'\mapsto 0
    \})\\
    \justification{Logic} 
    &f(t)= f(t_0)\cup \{w\}\\ 
    \justification{Definition of $addword^{Dict}$ and $addword^{DictA}$ }
    &post_{addword^{Dict}}\subst{f(t_0),f(t)}{W_0, W}
\end{align*}
For the $checkword$ we proof:
\begin{align*}
    &pre_{checkword^{Dict}}\subst{f(t_0)}{W}\And post_{checkword^{DictA}}\\
    \justification{Definition of $checkword^{DictA}$ and $checkword^{Dict}$}
    &\True\subst{f(t_0)}{W} \And b = (w\in \domt\And t = t_0 )\\
    %
    \justification[\Implies]{Definition of $f$}
    &b = (w \in f(t))\And t = t_0\\
    \justification{Definition of $checkword^{DictA}$ and $checkword^{Dict}$}
    & post_{checkword^{Dict}}\subst{f(t_0),f(t)}{W_0,W}
\end{align*}
For the $delword$ we proof:
\begin{align*}
    &pre_{delword^{Dict}}\subst{f(t_0)}{W}\And post_{delword^{DictA}}\\
    \justification{Definition of $delword^{DictA}$ and $delword^{Dict}$}
    &w\in f(t_0) \And b = b_0\And
        (w \notin \domt \OR t\Ass t: w \mapsto 0)\\
    \justification[\Implies]{Definition of $f$}
    &f(t) = f(t_0)\backslash\{w\}\\
    \justification{Definition of $delword^{DictA}$ and $delword^{Dict}$}
    &post_{delword^{Dict}}\subst{f(t_0),f(t)}{W_0,W}
\end{align*}

\section{Task 3}
We derive our code in to fours part by $init$ and its operations. 
\subsection{init}
From the spec we have:
\begin{align*}
    &\domt =\{\epsilon\} \And f(t)= \phi\\
    \refstep{ass}{t \Ass \{\epsilon \mapsto 0\}}
\end{align*}

\subsection{addword}
From the spec we have:
\begin{gather*}
    \PROC\ addword^{DictA}(\WORD w)\cdot \\
        \qquad   
        \nt{
            b, t:[\True, b = b_0\And t = t_0\cup \{w\mapsto 1\}\cup 
            \{
                w' < w \And w' \notin \domt | w' \mapsto 0
            \}]
        }{(A1)}
\end{gather*}
\begin{align*}
    \lrefstep{(A1)}{c-frame}{
        t:[\True,t = t_0\cup \{w\mapsto 1\}\cup 
        \{
            w' < w \And w' \notin \domt | w' \mapsto 0
        \}]
    }
\end{align*}
\subsection{checkword}
From the spec we have:
\begin{gather*}
    \FUNC\ checkword^{DictA}(\WORD w)\\
        \qquad
        \nt{
            \mathbb{B}\cdot \VAR\ b\cdot b, t:
            [\True, b = (w\in \domt\And t = t_0)];\ \RETURN\ b
        }{(C1)}
\end{gather*}
\begin{align*}
    \lrefstep{(C1)}{c-frame}{
        \mathbb{B}\cdot \VAR\ b\cdot b:
        [\True, b = (w\in \domt\And t = t_0)];\ \RETURN\ b
    }
\end{align*}
\subsection{delword}
From the spec we have:
\begin{gather*}
    \PROC\ delword^{DictA}(\WORD w)\\
        \qquad
        \nt{
            \cdot b, t:
            [\True , b = b_0\And
            (w \notin \domt \OR t\Ass t: w \mapsto 0)]
        }{(D1)}    
\end{gather*}


\end{document}