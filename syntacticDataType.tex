\documentclass[a4paper,12pt,fleqn]{scrartcl}
\usepackage[l2tabu,orthodox]{nag}% Old habits die hard. All the same, there are commands, classes and packages which are outdated and superseded. nag provides routines to warn the user about the use of those.

\usepackage[all,error]{onlyamsmath}% Error on deprecated math commands like $$ $$.
\usepackage[strict=true]{csquotes}

%\usepackage{color}

\usepackage{listings}
\lstset{frame=single,framerule=0pt,language={C},showstringspaces=false,numbers=left,columns=fullflexible}


% COMP2111-specific macros. See
% http://www.cse.unsw.edu.au/~cs2111/18s1/LaTeX/primer.html
\usepackage{2111defs2}
\usepackage[colorlinks=true]{hyperref}

\newcommand{\assn}[1]{{\color{red}\left\{#1\right\}}}
\newcommand{\remark}[1]{{\sffamily\color{blue}{#1}}}

% define some convenience macros specific to this task
\newcommand{\perm}{\mathsf{perm}}
%title variable
\title{Assignment 3}
\author{Ruofei HUANG(z5141448)\and
Anqi ZHU(z5141541)
}


\newcommand{\variant}[3]{(#1:#2\mapsto #3)}
\newcommand{\oldwhatever}[4]{\variant{#1}{#2}{\variant{#1[#2]}{#3}{#4}}}
\newcommand{\whatever}[4]{\variant{#1}{#2\mapsto #3}{#4}}
\newcommand{\ah}{\mathsf{a}}
\newcommand{\be}{\mathsf{b}}
\newcommand{\length}[1]{\left|#1\right|}
\newcommand{\noof}[2]{\left\|#1\right\|_{#2}}

\newcommand{\pre}{\mathit{pre}}
\newcommand{\post}{\mathit{post}}
\def\L{\mathcal{L}}
\begin{document}
\maketitle
\section{Task 1}

According to the specified problem statement in the assignment, 
we could describe the syntactic data type $Dict$ as below.
The encapsulated state is a dictionary word set $W$.
\begin{align*}
    &Dict = 
    (
        W = \phi,  \\
        &\left( 
            \begin{array}{l}
                \textbf{proc}\ addword^{Dict}(\textbf{value}\ w)\cdot 
                b, W:[ \ \True, b = b_0\And W = W_0\cup \{w\}] \ \\
                \textbf{func}\ checkword^{Dict}(\textbf{value}\ w):\mathbb{B}\cdot
                \textbf{var}\ b\cdot\ b, W:[ \ \True, b = (w\in W_0)];\ \textbf{return}\ b\\
                \textbf{proc}\ delword^{Dict}(\textbf{value}\ w)\cdot
                b, W:[ w\in W, b = b_0\And W = W_0 \backslash \{w\}]
            \end{array}
        \right)
    )
\end{align*}
\section{Task 2}
Now we would like to refine $Dict$ to a second data type $DictA$ where we replace $W$ with a trie $t$, 
the corresponding trie domain $\textbf{dom}t$ and 
a counter $i$ that holds the index of the next free cell in the domain t array.
We also borrow the definition of $\mathcal{T}$ in the problem statement.
It represents the set of all tries according to the domain. 
We shall use this definition later in our refinement.$\\$
/*probably need to define a data invariant here*/$\\$
This suggests we should first build up a inductively defined predicate to ensure the provable relations between
$DictA$ and $Dict$.
\begin{align*}
    % r = &\forall w\in W\ (\forall\epsilon\leq v\leq w.\ v\in \textbf{dom}t)\And\\
    %     &\left(
    %         \begin{array}{l}
    %             (W = \phi\Leftrightarrow t = \{\epsilon\mapsto 1\})\And \\
    %             (W\neq\phi\Leftrightarrow \forall v\in \textbf{dom}t.\ ((v\mapsto 1)\in t\And v\in W)\And((v\mapsto 0)\in t\And v\notin W))
    %         \end{array}    
    %      \right) 

    r = & ((\epsilon\mapsto 1)\in t\Or i = 0\Leftrightarrow W = \phi)\And\\
        & ((\epsilon\mapsto 0)\in t\And i >0\Leftrightarrow W = \exists S(W =\\
        \left(
            \begin{array}{l}
                (\textbuf{dom}t[i-1]}\cup S\And r\subst{S}{W}\subst{i-1}{i}\And t[i-1] = \textbuf{dom}t[i-1]\mapsto 1)\And\\
                (S\And r\subst{S}{W}\subst{i-1}{i}\And t[i-1] = \textbuf{dom}t[i-1]\mapsto 0)
            \end{array}
        \right) 
        ))
\end{align*}
which we can translate into a function from concrete to abstract values:
\begin{align*}
    f(t, i) = 
    \left\{
        \begin{array}{l}
            \phi & \text{if }(\epsilon\mapsto 1)\in t\Or i = 0\\
            \textbuf{dom}t[i-1]}\cup f(t, i-1) & \text{if }(\epsilon\mapsto 0)\in t\And i >0\And t[i-1] = \textbuf{dom}t[i-1]\mapsto 1\\
            f(t, i-1) & \text{otherwise }
        \end{array}
    \right
\end{align*}
With that in mind we propose the initialisation predicate $init^{DictA} = (i = 0)$ and operations given as follows.
\begin{align*}
    continue
\end{align*} 
\end{document}